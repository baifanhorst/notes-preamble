%%%%%%%%%%%%%%%%%%%%%%%%%%%%%%%%%%
% Derivatives
%%%%%%%%%%%%%%%%%%%%%%%%%%%%%%%%%%%

%differential symbol
\newcommand\dif{\mathrm{d}} 
% Partial derivative
\newcommand\pp[2]{\frac{\partial #1}{\partial #2}} 
% Partial derivative in text
\newcommand\pptext[2]{\partial #1 / \partial #2} 
% Derivative
\newcommand\dd[2]{\frac{\mathrm{d} #1}{\mathrm{d} #2}}
% Derivative in text
\newcommand\ddtext[2]{\mathrm{d} #1/\mathrm{d} #2} 
% Second derivative
\newcommand\ddt[2]{\frac{\mathrm{d}^2 #1}{\mathrm{d} {#2}^2}} 
% Third derivative
\newcommand\ddthr[2]{\frac{\mathrm{d}^3 #1}{\mathrm{d} {#2}^3}} 
% nth derivative
\newcommand\ddn[3]{\frac{\mathrm{d}^{#1} #2}{\mathrm{d} {#3}^{#1}}} 

% partial differential operator
\newcommand\pps[1]{\partial_{#1}} 
% Second partial derivative
\newcommand\ppt[2]{\frac{\partial^2 #1}{\partial {#2}^2}} 
% Second partial derivative in text
\newcommand\ppttext[2]{\partial^2 #1 / \partial {#2}^2}
% Second mixed partial derivative
\newcommand\pptm[3]{\frac{\partial^2 #1}{\partial #2 \partial #3}} 

% Third derivative
\newcommand\ppth[2]{\frac{\partial^3 #1}{\partial {#2}^3}} 
% Third mixed derivative 1-2
\newcommand\ppthot[3]{\frac{\partial^3 #1}{\partial {#2} \partial {#3}^2}} 
% Third mixed derivative 2-1
\newcommand\ppthto[3]{\frac{\partial^3 #1}{\partial {#2}^2 \partial {#3}}} 



% Fourth partial derivative
\newcommand\ppf[2]{\frac{\partial^4 #1}{\partial {#2}^4}} 
% Mixed fourth derivative 2-2
\newcommand\ppftt[3]{\frac{\partial^4 #1}{\partial {#2}^2 \partial {#3}^2}} 
% Mixed fourth derivative 1-3
\newcommand\ppfoth[3]{\frac{\partial^4 #1}{\partial {#2} \partial {#3}^3}} 
% Fourth order mixed derivative 3-1
\newcommand\ppftho[3]{\frac{\partial^4 #1}{\partial {#2}^3 \partial {#3}}} 


% Material derivative
\newcommand\DD[2]{\frac{\mathrm{D} #1}{\mathrm{D} #2}}
% Material derivative in text
\newcommand\DDtext[2]{\mathrm{D} #1/\mathrm{D} #2} 
% Second material derivative
\newcommand\DDt[2]{\frac{\mathrm{D}^2 #1}{\mathrm{D} {#2}^2}}
% Material derivative denoted by a dot
\newcommand\DDdot[1]{\dot{#1}}


% Inexact differential
\def\dbar{{\mathchar'26\mkern-12mu \mathrm{d}}}







%%%%%%%%%%%%%%%%%%%%%%%%%%%%%%%%%%
% Operators
%%%%%%%%%%%%%%%%%%%%%%%%%%%%%%%%
% Absolute value
\DeclarePairedDelimiter{\abs}{\lvert}{\rvert} 
% norm
\DeclarePairedDelimiter{\norm}{\lVert}{\rVert} 
% det
\DeclarePairedDelimiter{\mydet}{\lvert}{\rvert} 
% inner product 1
\DeclarePairedDelimiter{\inner}{\langle}{\rangle} 
% inner product 2
\newcommand\innerp[2]{\langle #1,#2\rangle} 
% Trace
\DeclareMathOperator{\tr}{tr} 
% Rank
\newcommand\rank{\mathrm{rank}} 
% Span
\newcommand\myspan{\mathrm{span}} 
% Diameter
\newcommand\diam{\mathrm{diam}} 
% Complex conjugate
\newcommand\conj[1]{\overline{#1}} 
% ???
%\newcommand\bc[1]{\overline{#1}} 
% Sign
\newcommand\sign{\mathrm{sign}} 
% Vector
\newcommand\vect[1]{\underline{#1}}
% Bold font vector
\newcommand\vb[1]{\bm{#1}} 
% Tensor
\newcommand\mytensor[1]{\tilde{#1}}
% Mean
\newcommand\mean[1]{\langle #1\rangle}
% Infinity mean
\newcommand\meaninf[1]{{\langle #1\rangle}_{\infty}}

%\newcommand\matr[1]{\uwave{#1}} %matrix

% Probability
\newcommand\mypr[1]{\Pr\{ #1 \}}
% Probability
\newcommand\Prob[1]{\text{Pr}\{ #1 \}}

%%%%%%%%%%%%%%%%%%%%%%%%%
% Symbols
%%%%%%%%%%%%%%%%%%%%%%%%%

%Roman numbers 
\newcommand{\Rnum}[1]{\uppercase\expandafter{\romannumeral #1\relax}} 
% Set braces
\newcommand\myset[1]{\{ #1 \}}
% Set symbols
\newcommand\Cset{\mathbb{C}}
\newcommand\Rset{\mathbb{R}}
\newcommand\Nset{\mathbb{N}}
\newcommand\Kset{\mathbb{K}}
\newcommand\Qset{\mathbb{Q}}
\newcommand\Fset{\mathbb{F}}
\newcommand\Eset{\mathbb{E}}
\newcommand\Zset{\mathbb{Z}}
%\newcommand\set[1]{\mathbb{#1}}

% Space symbol (such as Hilbert, Banach...)
\newcommand\mathspace[1]{\mathcal{#1}} % Math space


% Operator symbol, mostly used in Qiu's linear algebra course
\newcommand\oper[1]{\underline{#1}}
% Image
\newcommand\image{\mathrm{Im}}
% Null space
\newcommand\nullspace{\mathrm{Null}}
% Range
\newcommand\range{\mathrm{Range}}
% Convex hull
\newcommand\conv[1]{\mathrm{Conv}(#1)} 
% Interior of a set
\newcommand\Int[1]{\mathrm{Int}(#1)} 
% Closure of a set
\newcommand\Cl[2][]{\mathrm{Cl}_{#1}(#2)} 
% Boundary of a set
\newcommand\Bd[1]{\mathrm{Bd}(#1)} 
% Affine hull
\newcommand\Aff[1]{\mathrm{Aff}(#1)} 
% Kernel
\newcommand\Ker[1]{\mathrm{Ker}(#1)} 
% Topology
\newcommand\topo[1]{\mathcal{#1}} 
% Relative interior
\newcommand\Ri[1]{\mathrm{Ri}(#1)} 
% The graph of a function
\newcommand\Gf[1]{\mathrm{G}(#1)} 
% The epigraph of a function
\newcommand\Epi[1]{\mathrm{Epi}(#1)} 
% The hypograph of a function
\newcommand\Hyp[1]{\mathrm{Hyp}(#1)} 
% Real part of a complex number
\renewcommand\Re[1]{\mathrm{Re}(#1)}
% Imaginary part of a complex number
\renewcommand\Im[1]{\mathrm{Im}(#1)}
% Argmin
\DeclareMathOperator*{\argmin}{argmin}
% Argmax
\DeclareMathOperator*{\argmax}{argmax}
% Diameter of a set
\newcommand\diag{\mathrm{diag}}


% Operator L
\newcommand\Lop{\mathcal{L}}
% Operator N
\newcommand\Nop{\mathcal{N}}

% Laplace transformed function
\newcommand\Laplace[1]{\bar{#1}}
% Fourier transformed function
\newcommand\Fourier[1]{\tilde{#1}}



% Inhibitory arrow
\DeclareFontFamily{U}{FdSymbolC}{}
\DeclareFontShape{U}{FdSymbolC}{m}{n}{<-> s * FdSymbolC-Book}{}
\DeclareSymbolFont{fdarrows}{U}{FdSymbolC}{m}{n}
\DeclareMathSymbol{\leftfootline}{\mathrel}{fdarrows}{"AC}
\DeclareMathSymbol{\rightfootline}{\mathrel}{fdarrows}{"AD}
\DeclareMathSymbol{\longleftfootline}{\mathrel}{fdarrows}{"C6}
\DeclareMathSymbol{\longrightfootline}{\mathrel}{fdarrows}{"C7}


% Text: Schwarz-Christoffel 
\newcommand\schzchris{Schwarz-Christoffel\ }


















