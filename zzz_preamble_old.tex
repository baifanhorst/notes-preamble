%%%%%%%%%%%%%%%%%%%%%%%%%%%%
% General style
%%%%%%%%%%%%%%%%%%%%%%%%%%%%%

%\usepackage{graphicx}

\usepackage{geometry} %page margins
	\geometry{a4paper,left=1.5cm,right=1.5cm,top=1.5cm,bottom=1.5cm}
	
	
%\usepackage{setspace} %double space
%\doublespacing



%使用插图标题
\usepackage{caption,subcaption} 

%%%%%%%%%%%%%%%%%%%%%%%%%%%%%%%%%%%%%
% Table of Contents, List of Figures and Tables
%%%%%%%%%%%%%%%%%%%%%%%%%%%%%%%%%%%%%%%%
\usepackage{tocloft}
\renewcommand{\cftfigpresnum}{Figure\space} % <-- Add prefix "Figure "
\setlength{\cftfigindent}{0pt} % <-- Remove indentation
\setlength{\cftfignumwidth}{60pt} % <-- Set width for the number column

\renewcommand{\cfttabpresnum}{Table\space} % <-- Add prefix "Table "
\setlength{\cfttabindent}{0pt}            % <-- Remove indentation
\setlength{\cfttabnumwidth}{60pt} % <-- Set width for the number column


%%%%%%%%%%%%%%%%%%%%%%%%%%%%%%%%%%
% General Math Settings
%%%%%%%%%%%%%%%%%%%%%%%%%%%%%%%%%%
\usepackage{amsmath}
\usepackage{amssymb}
\usepackage{bm} %bold font for letters in equations
\usepackage{amsfonts} %sets of real numbers, integers...
\usepackage{mathtools} %To use definition symbol \coloneqq and \eqqcolon



%%%%%%%%%%%%%%%%%%%%%%%%%%%%%%%%%%
% Derivatives
%%%%%%%%%%%%%%%%%%%%%%%%%%%%%%%%%%%

\newcommand\dif{\mathrm{d}} %differential symbol
\newcommand\pp[2]{\frac{\partial #1}{\partial #2}} %Partial derivatives
\newcommand\pptext[2]{\partial #1 / \partial #2} %Partial derivatives in text
\newcommand\dd[2]{\frac{\mathrm{d} #1}{\mathrm{d} #2}}
\newcommand\ddtext[2]{\mathrm{d} #1/\mathrm{d} #2} %行文中的导数
\newcommand\ddt[2]{\frac{\mathrm{d}^2 #1}{\mathrm{d} {#2}^2}} %二次导数
\newcommand\ddthr[2]{\frac{\mathrm{d}^3 #1}{\mathrm{d} {#2}^3}} %三次导数
\newcommand\ddn[3]{\frac{\mathrm{d}^{#1} #2}{\mathrm{d} {#3}^{#1}}} % nth derivative
\newcommand\pps[1]{\partial_{#1}} % partial differential operator
\newcommand\ppt[2]{\frac{\partial^2 #1}{\partial {#2}^2}} %二次偏导数
\newcommand\pptm[3]{\frac{\partial^2 #1}{\partial #2 \partial #3}} %二次混合偏导数
\newcommand\ppf[2]{\frac{\partial^4 #1}{\partial {#2}^4}} % fourth order derivative
\newcommand\ppftt[3]{\frac{\partial^4 #1}{\partial {#2}^2 \partial {#3}^2}} % Fourth order mixed derivative pf_pxx_pyy
\newcommand\ppfoth[3]{\frac{\partial^4 #1}{\partial {#2} \partial {#3}^3}} % Fourth order mixed derivative pf_px_pyyy
\newcommand\ppftho[3]{\frac{\partial^4 #1}{\partial {#2} \partial {#3}^3}} % Fourth order mixed derivative pf_pxxx_py

\newcommand\ppth[2]{\frac{\partial^3 #1}{\partial {#2}^3}} % 3rd order derivative
\newcommand\ppthot[3]{\frac{\partial^3 #1}{\partial {#2} \partial {#3}^2}} % Fourth order mixed derivative pf_px_pyyy
\newcommand\ppthto[3]{\frac{\partial^3 #1}{\partial {#2}^2 \partial {#3}}} % Fourth order mixed derivative pf_px_pyyy



\newcommand\DD[2]{\frac{\mathrm{D} #1}{\mathrm{D} #2}}
\newcommand\DDt[2]{\frac{\mathrm{D}^2 #1}{\mathrm{D} {#2}^2}} %二次导数
\newcommand\DDtext[2]{\mathrm{D} #1/\mathrm{D} #2} %行文中的导数



%%%%%%%%%%%%%%%%%%%%%%%%%%%%%%%%%%
% Operators
%%%%%%%%%%%%%%%%%%%%%%%%%%%%%%%%
\DeclarePairedDelimiter{\abs}{\lvert}{\rvert} % Absolute value
\DeclarePairedDelimiter{\norm}{\lVert}{\rVert} % norm
\DeclarePairedDelimiter{\mydet}{\lvert}{\rvert} % det
\DeclarePairedDelimiter{\inner}{\langle}{\rangle} % inner product
\newcommand\innerp[2]{\langle #1,#2\rangle} % inner product
\DeclareMathOperator{\tr}{tr} % Trace
\newcommand\rank{\mathrm{rank}} % Rank
\newcommand\myspan{\mathrm{span}} % Span
\newcommand\diam{\mathrm{diam}} % Diameter
\newcommand\conj[1]{\overline{#1}} % Complex conjugate}
\newcommand\bc[1]{\overline{#1}} 
\newcommand\sign{\mathrm{sign}} 

\newcommand\vect[1]{\underline{#1}} %vector
\newcommand\vb[1]{\bm{#1}} % Bold font vector
\usepackage{esvect} % Vector arrow \vv
\newcommand\mytensor[1]{\tilde{#1}} % Tensor notation


\usepackage{ulem} %为了加入下波浪线
\newcommand\matr[1]{\uwave{#1}} %matrix
\newcommand\mean[1]{\langle #1\rangle}
\newcommand\meaninf[1]{{\langle #1\rangle}_{\infty}}



\newcommand{\Rnum}[1]{\uppercase\expandafter{\romannumeral #1\relax}} %Roman numbers 

\newcommand\myset[1]{\{ #1 \}}
\newcommand\mypr[1]{\Pr\{ #1 \}}
\newcommand\Cset{\mathbb{C}}
\newcommand\Rset{\mathbb{R}}
\newcommand\Nset{\mathbb{N}}
\newcommand\Kset{\mathbb{K}}
\newcommand\Qset{\mathbb{Q}}
\newcommand\Fset{\mathbb{F}}
\newcommand\Eset{\mathbb{E}}
\newcommand\Zset{\mathbb{Z}}
\newcommand\set[1]{\mathbb{#1}}

\newcommand\mathspace[1]{\mathcal{#1}} % Math space



\newcommand\oper[1]{\underline{#1}}
\newcommand\image{\mathrm{Im}}
\newcommand\nullspace{\mathrm{Null}}
\newcommand\range{\mathrm{Range}}
\newcommand\conv[1]{\mathrm{Conv}(#1)} % Convex hull of a set
\newcommand\Int[1]{\mathrm{Int}(#1)} % Interior of a set
\newcommand\Cl[2][]{\mathrm{Cl}_{#1}(#2)} % Closure of a set
\newcommand\Bd[1]{\mathrm{Bd}(#1)} % Boundary of a set
\newcommand\Aff[1]{\mathrm{Aff}(#1)} % Affine hull
\newcommand\Ker[1]{\mathrm{Ker}(#1)} % Kernel
\newcommand\topo[1]{\mathcal{#1}} % Topology
\newcommand\Ri[1]{\mathrm{Ri}(#1)} % Relative interior
\newcommand\Gf[1]{\mathrm{G}(#1)} % The graph of a function
\newcommand\Epi[1]{\mathrm{Epi}(#1)} % The epigraph of a function
\newcommand\Hyp[1]{\mathrm{Hyp}(#1)} % The hypograph of a function

\renewcommand\Re[1]{\mathrm{Re}(#1)}
\renewcommand\Im[1]{\mathrm{Im}(#1)}

\DeclareMathOperator*{\argmin}{argmin}
\DeclareMathOperator*{\argmax}{argmax}

\newcommand\diag{\mathrm{diag}}

\newcommand\Prob[1]{\text{Pr}\{ #1 \}}


\newcommand\Lop{\mathcal{L}}
\newcommand\Nop{\mathcal{N}}


\usepackage{braket} % Dirac's notation

\newcommand\Laplace[1]{\bar{#1}}
\newcommand\Fourier[1]{\tilde{#1}}





%%%%%%%%%%%%%%%%%%%%%%%%%%%%%%%%%%%%%%%%%%
% Reference and Citation
%%%%%%%%%%%%%%%%%%%%%%%%%%%%%%%%%%%%%%%%%
\usepackage{hyperref}
\usepackage{cleveref}




\newcommand\myrefeq[1]{Eq.\,(\ref{#1})} %Eq. ()
\newcommand\myrefequation[1]{Equation\,(\ref{#1})} %Equation ()
\newcommand\myrefeqr[2]{Eqs.\,(\ref{#1})--(\ref{#2})} %Eq. ()
\newcommand\myrefeqt[2]{Eqs.\,(\ref{#1}) and (\ref{#2})} %Eq. () and ()
\newcommand\myrefeqth[3]{Eqs.\,(\ref{#1}), (\ref{#2}) and (\ref{#3})}
\newcommand\myrefeqf[4]{Eqs.\,(\ref{#1}), (\ref{#2}),  (\ref{#3}) and (\ref{#4})}

\newcommand\myrefequationth[3]{Equations\,(\ref{#1}), (\ref{#2}) and (\ref{#3})}


\newcommand\myreffig[1]{Fig.\,\ref{#1}} %Fig. #
\newcommand\myreffigure[1]{Figure\,\ref{#1}}
\newcommand\myreffigt[2]{Fig.\,\ref{#1} and \ref{#2}}
\newcommand\myreffiguret[2]{Figure\,\ref{#1} and \ref{#2}}
\newcommand\myreffigureth[3]{Figure\,\ref{#1}, \ref{#2} and \ref{#3}}
\newcommand\myreffigr[2]{Fig.\,\ref{#1} - \ref{#2}}
\newcommand\myrefsec[1]{Section\,(\ref{#1})}
\newcommand\myrefsubsec[1]{Subsection\,(\ref{#1})}
\newcommand\myrefappendix[1]{Appendix\,(\ref{#1})}

\newcommand\myrefchp[1]{Ch.\,\ref{#1}}
\newcommand\myrefchapter[1]{Chapter\,\ref{#1}}

\newcommand\mycite[1]{Ref.\,\cite{#1}}
\newcommand\mycitet[2]{Refs.\,\cite{#1} and \cite{#2}}

\newcommand\myreftb[1]{Table \ref{#1}}

\newcommand\myrefthm[1]{Thm.\,\ref{#1}} %Thm. #
\newcommand\myreflemma[1]{Lemma.\,\ref{#1}} %Lemma. #
\newcommand\myrefdef[1]{Def.\,\ref{#1}} %Def. #
\newcommand\myrefcoro[1]{Coro.\,\ref{#1}} %Coro. #

\newcommand\myrefitem[1]{item \ref{#1}}
\newcommand\myrefItem[1]{Item \ref{#1}}
\newcommand\myrefitemp[1]{(\ref{#1})}



\usepackage[numbers, sort&compress]{natbib}




\usepackage{float} %固定图片位置





\usepackage{longtable} %For a list of symbols





% Inhibitory arrow
\DeclareFontFamily{U}{FdSymbolC}{}
\DeclareFontShape{U}{FdSymbolC}{m}{n}{<-> s * FdSymbolC-Book}{}
\DeclareSymbolFont{fdarrows}{U}{FdSymbolC}{m}{n}
\DeclareMathSymbol{\leftfootline}{\mathrel}{fdarrows}{"AC}
\DeclareMathSymbol{\rightfootline}{\mathrel}{fdarrows}{"AD}
\DeclareMathSymbol{\longleftfootline}{\mathrel}{fdarrows}{"C6}
\DeclareMathSymbol{\longrightfootline}{\mathrel}{fdarrows}{"C7}



\usepackage{extarrows} %Used for an equal sign with text above it.

\usepackage{booktabs} %For three-line tables

% Enhanced formatting for author and affiliation blocks
\usepackage{authblk}


\usepackage{gensymb} %For the symbol of celcius degree \degree

\usepackage{listings} % For adding codes



%\usepackage{chemarrow} %使用双向箭头


%%%%%%%%%%%%%%%%%%%%%%%%%%%%%%%%%%%%%%%%%%
% Environment
%%%%%%%%%%%%%%%%%%%%%%%%%%%%%%%%%%%%%%%%%%
\usepackage{enumitem} %制定列表可选参数

%自定义环境
\newenvironment{myitemone}[1]{\par \noindent {\bfseries #1 }\par}{\vspace{2cm}\par}


%%%%%%%%%%%%%%%%%%%%%%%%%%%%%%%%%%%%5
% Theorem-like environments
\usepackage{amsthm} % \newtheorem* requires this package

%\newtheorem{thm}{Theorem}
%\newtheorem*{remark}{Remark}
%\newtheorem{mydef}[thm]{Definition}
%\newtheorem{lemma}[thm]{Lemma}
%\newtheorem{example}[thm]{Example}
%\newtheorem{coro}[thm]{Corollary}
%\newtheorem{alg}[thm]{Algorithm}
%\newtheorem*{myproof}{Proof}
%\newtheorem*{summary}{Summary}
%\newtheorem*{solution}{Solution}



%%%%%%%%%%%%%%%%%%%%%%%%%%%%%%%%%%%%%%%%
% Color-boxed theorem environment
\usepackage[most]{tcolorbox} % Color boxes for theorem-like environments
%\definecolor{headlinecolor}{RGB}{220,230,241}
%\definecolor{contentcolor}{RGB}{255,255,224}
%\definecolor{bordercolor}{RGB}{100,149,237}
\tcbuselibrary{breakable} % Make a box span multiple pages

% Initialize a new counter for theorem numbering
\newcounter{thm}
% Renew \thetheorem to specify how a theorem number will be displayed.
% The following sets global numbering, irrespective of section number
% If we want to involve section number, we could use:
% \renewcommand{\thetheorem}{\thesubsection.\arabic{theorem}}
% But this is not flexible. For example, if a theorem is in not in a subsection, 
% the number may not be shown correctly.
\renewcommand{\thethm}{\arabic{thm}}


% Theorem color box

\definecolor{mycolorback}{RGB}{224, 224, 224}       % (255/255, 64/255, 0/255)
\definecolor{mycolorframe}{RGB}{46, 38, 38} % (255/255, 218/255, 185/255)




\newtcolorbox{thmbox}[1][]{
	breakable, % Allows the box to span multiple pages
	colback=mycolorback,
	colframe=mycolorframe,
	%colback=black!5!white, % Background color specified in xcolor style, meaning 5% black and 95% white
	%colframe=black!75!white, % Frame color
	fonttitle=\bfseries,
	title={Theorem}, % We'll add the number in the environment definition
	#1 % Command provided by the optional argument
}

% Theorem environment

\newenvironment{thm}[1][]{
	\refstepcounter{thm} % Increment and make the counter referencable
	\begin{thmbox}[title={Theorem \thethm\IfNoValueF{#1}{ #1}}] % Display the theorem number and optional title
	}{\end{thmbox}}






% Definition environment
\newenvironment{mydef}[1][]{
	\refstepcounter{thm} % Increment and make the counter referencable
	\begin{thmbox}[title={Definition \thethm\IfNoValueF{#1}{ #1}}] % Display the theorem number and optional title
	}{\end{thmbox}}


% Lemma environment
\newenvironment{lemma}[1][]{
	\refstepcounter{thm} % Increment and make the counter referencable
	\begin{thmbox}[title={Lemma \thethm\IfNoValueF{#1}{ #1}}] % Display the theorem number and optional title
	}{\end{thmbox}}


% Corollary environment
\newenvironment{coro}[1][]{
	\refstepcounter{thm} % Increment and make the counter referencable
	\begin{thmbox}[title={Corollary \thethm\IfNoValueF{#1}{ #1}}] % Display the theorem number and optional title
	}{\end{thmbox}}


% Algorithm environment
\newenvironment{alg}[1][]{
	\refstepcounter{thm} % Increment and make the counter referencable
	\begin{thmbox}[title={Algorithm \thethm\IfNoValueF{#1}{ #1}}] % Display the theorem number and optional title
	}{\end{thmbox}}


% Example environment
\newenvironment{example}[1][]{
	\refstepcounter{thm} % Increment and make the counter referencable
	\begin{thmbox}[title={Example \thethm\IfNoValueF{#1}{ #1}}] % Display the theorem number and optional title
	}{\end{thmbox}}


% Remark environment
\newenvironment{remark}[1][]{
	\refstepcounter{thm} % Increment and make the counter referencable
	\begin{thmbox}[title={Remark \thethm\IfNoValueF{#1}{ #1}}] % Display the theorem number and optional title
	}{\end{thmbox}}

\newenvironment{myproof}[1][]{
	\begin{thmbox}[title={Proof {#1}}] % Display the theorem number and optional title
	}{\end{thmbox}}


%\newtcolorbox{myproof}
%{breakable, % Make the box span multiple pages
	% colback=black!5!white,
	% colframe=black!75!white,
	% fonttitle=\bfseries,
	% title=Proof}

\newtcolorbox{summary}
{breakable, 
	colback=mycolorback,
	colframe=mycolorframe,
	fonttitle=\bfseries,
	title=Summary}

\newtcolorbox{solution}
{breakable, 
	colback=mycolorback,
	colframe=mycolorframe,
	fonttitle=\bfseries,
	title=Solution}






%%%%%%%%%%%%%%%%%%
% Only box, without title, used to highlight equations
\newtcolorbox{mytcb}{
	breakable, 
	colback=mycolorback,
	colframe=mycolorframe,}
% With title bar only
\newtcolorbox{mytcbt}[1][]
{breakable, 
	colback=mycolorback,
	colframe=mycolorframe,
	fonttitle=\bfseries,
	title=\protecting{#1}}
% In the title definition, \protecting is needed to ensure that 
% math formulas can be used in the title.




\newcommand\enumlabel{(\arabic*)} % Make item labels (1), (2),... in enumerate environment
\newcommand\enumlabelproof{[[\arabic*]]} % Item labels for enumerate enviroment in a proof.







%%%%%%%%%%%%%%%%%%%%%%%%%%%%%%%%%%%
% Color
%%%%%%%%%%%%%%%%%%%%%%%%%%%%%%%%%%%
\usepackage{color}

% For color \color{red}
% [table] enables coloring of tables
\usepackage[table]{xcolor} 
\definecolor{protecteyes}{rgb}{0.9, 0.99, 0.9}

\newcommand{\red}[1]{\textcolor{red}{#1}}


%%%%%%%%%%%%%%%%%%%%%%%%%%%%%%%%%%%%%
% Drawing
%%%%%%%%%%%%%%%%%%%%%%%%%%%%%%%%%%
\usepackage{tikz}  








%%%%%%%%%%%%%%%%%%%%%%%%%%%%%%%%%%
% Misc
%%%%%%%%%%%%%%%%%%%%%%%%%%%%%%%

% Check mark and cross mark 
\usepackage{pifont}
\newcommand{\cmark}{\ding{51}}
\newcommand{\xmark}{\ding{55}}


\usepackage{empheq}

\usepackage{algorithm}
\usepackage{algpseudocode} % For writing codes


\newcommand\newpara{\vspace{1cm}\noindent} % Begin a new paragraph
\newcommand\wrt{with respect to\ }

















