\usepackage{tensor} % For tensor indices

\newcommand\gco[1]{\bm{g_{#1}}} % Covariant basis
\newcommand\gct[1]{\bm{g^{#1}}} % Contravariant basis
\newcommand\gcoL[1]{ \bm{\hat{g}_{#1}} } % Covariant basis wrt Lagrange coordinates at t
\newcommand\gctL[1]{ \bm{\hat{g}^{#1}} } % Contravariant basis wrt Lagrange coordinates at t
\newcommand\gcoLi[1]{ \bm{\mathring{g}_{#1}} } % Covariant basis wrt Lagrange coordinates at t=0
\newcommand\gctLi[1]{ \bm{\mathring{g}^{#1}} } % Contravariant basis wrt Lagrange coordinates at t=0

\newcommand\gradL{ \hat{\nabla} } % Gradient wrt Lagrange coordinates
\newcommand\gmetL[2]{\hat{g}_{#1 #2}} % metric wrt Lagrange coordinates

\newcommand\invar[2]{\mathcal{#1}_{#2}} %Invariant of a 2nd order tensor

\newcommand\twovstars{\tensor*{}{*^{*}_{*}}} %Vertical double stars

\newcommand\dcov[2]{{#1}_{;#2}} % Covariant derivative


\newcommand\depszero{  
	\frac{\mathrm{d}}{\mathrm{d} \epsilon}\Big|_{\epsilon=0} }

\newcommand\Vinit{\mathring{V}} % volumn in the initial config
\newcommand\ainit{\mathring{a}} % area in the initial config
\newcommand\Ainit{\mathring{A}} % area in the initial config
\newcommand\rhoinit{\mathring{\rho}}
\newcommand\uinit{\mathring{u}}
\newcommand\vinit{\mathring{u}}
\newcommand\GAinit{\mathring{\Gamma}} % boundary in the initial config, can be used for both 2D and 3D cases.
\newcommand\IC[1]{\mathring{#1}} % A quantity in the initial config


%%%%%%%%%%%%%%%%%%%%%%%%%%%%%%%
% Notation used in the weak form
%%%%%%%%%%%%%%%%%%%%%%%%%%%%%%%%
\newcommand\dm{\Omega} % Current domain
\newcommand\dminit{\mathring{\Omega}} % Initial domain
\newcommand\bdry{\Gamma} % Current boundary
\newcommand\bdryinit{\mathring{\Gamma}} % Initial boundary
\newcommand\bdryu{\Gamma_u} % Current displacement boundary
\newcommand\bdryt{\Gamma_t} % Current traction boundary
\newcommand\bdrytinit{\mathring{\Gamma}_t} % Initial traction boundary
\newcommand\bdrycinit{\mathring{\Gamma}_c} % Initial contact boundary
\newcommand\bdrycsinit{\mathring{\Gamma}_c^{(s)}}
\newcommand\bdrycminit{\mathring{\Gamma}_c^{(m)}}

\newcommand\deps{\frac{\mathrm{d}}{\mathrm{d} \epsilon} \Bigg|_{\epsilon=0} } % Used in the definition of directional derivative.



\newcommand\dSdE{\vb{C^4}}% The derivative of S wrt E. In bonet's book this is denoted by C
\newcommand\dSdEcomp{(C^4)} % Used for the component of \dSdE

\newcommand\cfour{\vb{c^4}}% the 4th tensor related to dSdE
\newcommand\cfourcomp{(c^4)} % component of it


\newcommand\Wint{W_{\text{int}}}
\newcommand\Wext{W_{\text{ext}}}
\newcommand\Wb{W_b}
\newcommand\Wt{W_t}
\newcommand\Wc{W_c}
\newcommand\Fint{F^{\text{int}}}