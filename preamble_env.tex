% Item environment, used for reviewer's comments
\newenvironment{myitemone}[1]{\par \noindent {\bfseries #1 }\par}{\vspace{2cm}\par}


%%%%%%%%%%%%%%%%%%%%%%%%%%%%%%%%%%%%5


%\newtheorem{thm}{Theorem}
%\newtheorem*{remark}{Remark}
%\newtheorem{mydef}[thm]{Definition}
%\newtheorem{lemma}[thm]{Lemma}
%\newtheorem{example}[thm]{Example}
%\newtheorem{coro}[thm]{Corollary}
%\newtheorem{alg}[thm]{Algorithm}
%\newtheorem*{myproof}{Proof}
%\newtheorem*{summary}{Summary}
%\newtheorem*{solution}{Solution}



%%%%%%%%%%%%%%%%%%%%%%%%%%%%%%%%%%%%%%%%

%\definecolor{headlinecolor}{RGB}{220,230,241}
%\definecolor{contentcolor}{RGB}{255,255,224}
%\definecolor{bordercolor}{RGB}{100,149,237}
\tcbuselibrary{breakable} % Make a box span multiple pages

% Initialize a new counter for theorem numbering
\newcounter{thm}
% Renew \thetheorem to specify how a theorem number will be displayed.
% The following sets global numbering, irrespective of section number
% If we want to involve section number, we could use:
% \renewcommand{\thetheorem}{\thesubsection.\arabic{theorem}}
% But this is not flexible. For example, if a theorem is in not in a subsection, 
% the number may not be shown correctly.
\renewcommand{\thethm}{\arabic{thm}}


% Theorem color box

\definecolor{mycolorback}{RGB}{224, 224, 224}       % (255/255, 64/255, 0/255)
\definecolor{mycolorframe}{RGB}{46, 38, 38} % (255/255, 218/255, 185/255)
\definecolor{mycolorframeproof}{RGB}{76, 153, 0} 




\newtcolorbox{thmbox}[1][]{
	breakable, % Allows the box to span multiple pages
	colback=mycolorback, % background color
	colframe=mycolorframe, % frame color
	%colback=black!5!white, % Background color specified in xcolor style, meaning 5% black and 95% white
	%colframe=black!75!white, % Frame color
	fonttitle=\bfseries, % title font bold
	title={Theorem}, % We'll add the number in the environment definition
	#1 % Command provided by the optional argument, often used to overwrite the settings above, such as using a different color and using a different title
}

% Theorem environment

\newenvironment{thm}[1][]{
	\refstepcounter{thm} % Increment and make the counter referencable
	\begin{thmbox}[title={Theorem \thethm\IfNoValueF{#1}{ #1}}] % Display the theorem number and optional title
	}{\end{thmbox}}






% Definition environment
\newenvironment{mydef}[1][]{
	\refstepcounter{thm} % Increment and make the counter referencable
	\begin{thmbox}[title={Definition \thethm\IfNoValueF{#1}{ #1}}] % Display the theorem number and optional title
	}{\end{thmbox}}


% Lemma environment
\newenvironment{lemma}[1][]{
	\refstepcounter{thm} % Increment and make the counter referencable
	\begin{thmbox}[title={Lemma \thethm\IfNoValueF{#1}{ #1}}] % Display the theorem number and optional title
	}{\end{thmbox}}


% Corollary environment
\newenvironment{coro}[1][]{
	\refstepcounter{thm} % Increment and make the counter referencable
	\begin{thmbox}[title={Corollary \thethm\IfNoValueF{#1}{ #1}}] % Display the theorem number and optional title
	}{\end{thmbox}}


% Algorithm environment
\newenvironment{alg}[1][]{
	\refstepcounter{thm} % Increment and make the counter referencable
	\begin{thmbox}[title={Algorithm \thethm\IfNoValueF{#1}{ #1}}] % Display the theorem number and optional title
	}{\end{thmbox}}


% Example environment
\newenvironment{example}[1][]{
	\refstepcounter{thm} % Increment and make the counter referencable
	\begin{thmbox}[title={Example \thethm\IfNoValueF{#1}{ #1}}] % Display the theorem number and optional title
	}{\end{thmbox}}


% Remark environment
\newenvironment{remark}[1][]{
	\refstepcounter{thm} % Increment and make the counter referencable
	\begin{thmbox}[title={Remark \thethm\IfNoValueF{#1}{ #1}}] % Display the theorem number and optional title
	}{\end{thmbox}}

\newenvironment{myproof}[1][]{%
	\begin{thmbox}[title={Proof {#1}},%
		colframe=mycolorframeproof%
		] % Display the theorem number and optional title
	}{\end{thmbox}}


%\newtcolorbox{myproof}
%{breakable, % Make the box span multiple pages
	% colback=black!5!white,
	% colframe=black!75!white,
	% fonttitle=\bfseries,
	% title=Proof}

\newtcolorbox{summary}
{breakable, 
	colback=mycolorback,
	colframe=mycolorframe,
	fonttitle=\bfseries,
	title=Summary}

\newtcolorbox{solution}
{breakable, 
	colback=mycolorback,
	colframe=mycolorframe,
	fonttitle=\bfseries,
	title=Solution}






%%%%%%%%%%%%%%%%%%
% Only box, without title, used to highlight equations
\newtcolorbox{mytcb}{
	breakable, 
	colback=mycolorback,
	colframe=mycolorframe,}
% With title bar only
\newtcolorbox{mytcbt}[1][]
{breakable, 
	colback=mycolorback,
	colframe=mycolorframe,
	fonttitle=\bfseries,
	title=\protecting{#1}}
% In the title definition, \protecting is needed to ensure that 
% math formulas can be used in the title.

% Make item labels (1), (2),... in enumerate environment
\newcommand\enumlabel{(\arabic*)} 
% Item labels for enumerate enviroment in a proof.
\newcommand\enumlabelproof{[[\arabic*]]} 

